\chapter{Background}\label{c:background}

Several program analysis techniques have been proposed in the literature, in
order to aid developers and users discover interesting program properties within
their software. For example, one might be interested in finding out whether
there are memory leaks or whether some piece of code is reachable.

Such techniques come in different flavors, either static or dynamic, and are
usually based on strong mathematical concepts. In this section we provide
a brief background on these concepts and frameworks tailored to
the development of program analysis tools.

\section{Static Program Analysis}

Static program analysis is a program analysis technique that aims to reason
about a program's behaviors without actually executing it. Program analysis techniques have been
heavily utilized by optimizing compilers since their early stages and they
have also found several other applications among the areas of software security
and software correctness\cite{spa}. The question of whether
a program is correct or may terminate for all possible inputs is in general
undecidable. However, static analysis techniques are able to tackle
undecidability by over-approximating or under-approximating the initial problem,
attempting to reason over a simplified version of it.

There is a plethora of different static analysis algorithms in the literature, each one
met in several domains. For instance, one may utilize analyses such as \emph{liveness} and/or
\emph{pointer} analyses in an optimizing compiler with the intention of eliminating
dead code regions or performing a constant propagation/folding optimization.
Similarly, a \emph{reachability} analysis that determines whether a specific program point is
reachable could be used by a software correctness tool to make sure that
an erroneous state is actually never reached.

There is a variety of design choices that may prove essential towards the \emph{scalability}
and \emph{precision} of a static program analysis algorithm. We briefly describe some of
those choices in the context of this work.

\subsection{Whole-Program vs. Modular Analyses}

Whole-Program analyses should not be confused with \emph{inter-procedural} analyses.
An inter-procedural analysis considers multiple functions in order to perform
its computations. In the \emph{intra-procedural} setting instead, an analysis would
restrict its reasoning within a specific function bound, overlooking the way
that function calls or external dependencies affect the computations inside the function.
However, someone could also perform a modular-style analysis across a relatively small
set of functions, that is an \emph{inter-procedural} analysis.

A \emph{modular}-analysis performs its computations within modular
units, disregarding external code dependencies and mainly does not attempt
any reasoning accross the heap. On the contrary, a \emph{whole}-program analysis
reasons about a program's properties at every program point by
simultaneously constructing a whole-program heap representation of the program
to be leveraged during the analysis computations.

Whether the analysis would reason about a program's properties in a whole-program
or a modular manner is of great importance, because such a decision affects the
way that the analysis practitioner would tune its performance - scalability and
precision. In the whole-program analysis setting the design tradeoffs between
scalability and precision might prove crucial towards the overall analysis
reasoning, though a modular-analysis may be able to perform more precise reasoning,
due to the relatively restricted area of interest. Whole-program analyses are
commonly used in the analysis of languages with complex language features. Those
languages, such as the object-oriented ones need to reason about language constructs
that may reside on the heap, and thus they may benefit from a whole-program analysis
reasoning. Such an analysis may not take into consideration the
exact ordering of the instructions of a program, sacrificing precision towards
better scalability.

\subsection{Flow-Sensitivity}

The concept of whether an analysis is designed with respect to
the instruction ordering is called \emph{flow-sensitivity}.
On the contrary, an analysis that is not designed that way is
called flow-insensitive. Flow-sensitivity is tightly related with
the scalability and precision of static program analyses, due to the
fact that a flow-sensitive analysis keeps track of program properties
at every point of it, e.g. before or after every instruction.
Whole-program analyses may usually omit the control-flow constructs
during any of their reasoning. However, many whole-program static
analyses like the \doop{} framework's Pointer analysis for Java
manage to add flow-sensitivity to their core analysis by slight
preprocessing\cite{points-to15}.

Several tools utilize state-of-the-art compiler technologies to
convert the input source to a lower level \emph{intermediate-representation (IR)}.
For instance, there is a plethora of tools that utilize the \emph{LLVM Compiler 
Infrastructure}\cite{llvm-lattner} for \emph{C}-like languages,
or \emph{Soot - A Java optimization framework}\cite{soot} - for JVM languages.
These tools may further lower the source code in a
\emph{Static Single Assignment (SSA)} form. In SSA form, each variable is
assigned exactly once, never to be re-assigned again and it is also defined
before any of its uses, thus yielding a flow-sensitive representation.

\subsection{Path-Sensitivity}

A path-sensitive analysis is in a way an enhanced version of a
flow-sensitive analysis which also considers the path taken up to
a specific program point. Such an analysis
introduces a path representation usually encoded as the combination of
a program's neighboring branch conditional expressions,
named \textit{path predicate}. A path predicate is essentially a
\textit{Boolean formula}, that is a function of boolean variables whose assignment yields a
different control-flow path.

Explicit knowledge of the path taken up to a specific
program point allows the analysis to achieve higher
precision, ideally eliminating the need for any approximation.
However, such knowledge comes at a cost; the number of a program's
paths is exponential to the number of branches within the program,
thus an analysis that keeps track of all possible paths would not
manage to scale. There have been suggested multiple techniques
\cite{dillig2008sound}, \cite{das2002esp} to address such issues
in the literature. Some of those techniques manage to achieve
scalability by restricting the context of the computations or
by introducing loose abstractions of the initial problem.
As a static analysis technique a path-sensitive analysis does not
eventually execute the program, rather utilizes
the path encoding in order to assert properties that hold at
specific program points.

One of the adverse effects of pure Static Analysis techniques
is that they result in a large amount of \textit{false positives} by
trading off precision for scalability. In the context of
such techniques, a false positive is said to be an erroneous report
of a static analysis tool that a property violation has been
discovered within the analyzed program, though such violation
does not eventually exist. There have been
proposed several techniques of either (semi-) dynamic nature
that execute directly the program via code instrumentation, or
static solutions that attempt to simulate the execution of a program.

\section{Symbolic Execution}

Symbolic Execution is yet another program analysis technique that
may be of either static or dynamic flavor and mainly attempts
to answer whether certain properties of a program could potentially
be violated by a piece of code\cite{baldoni2018survey}. At the same
time it also results in an automated way to generate test cases for
programs under testing.

In contrast to the aforementioned static analysis techniques that
do not directly execute a program, Symbolic (or \textit{Concolic}) Execution
mainly attempts to simulate the execution of a
program by considering symbolic (non-deterministic) values for its input.
Usually, during the concrete execution of a program a single execution path is
considered for any computation, whilst a Symbolic Execution engine manages
to explore a plethora of paths thanks to the symbolic values assigned to
its input variables. As such, a concrete execution is said to under-approximate
the desired analysis. \textit{Concolic Execution} is a mixture of symbolic and concrete
execution that considers concrete input values when possible, thus making
symbolic execution feasible in practice. The latter is usually described
as Dynamic Symbolic Execution (DSE). Several tools have been proposed
in the literature such as SAGE\cite{godefroid2012sage} and KLEE\cite{cadar2008klee},
that are considered to be the tools of choice when it comes to
binary analysis and/or systems testing.

In the symbolic setting, execution is typically driven by a
\textit{symbolic execution engine} which at its core maintains some state
for every explored path. The core data-structures of such
an engine are the \textit{Boolean formulas} that encode an execution path by
considering the satisfied conditions taken for that exact path, and a
\textit{symbolic store} that keeps a mapping between the variables met within the path
and their equivalent symbolic expressions or values. The former are constructed
on each branch execution, while the latter is updated after an assignment to
the corresponding variable. Established by those two structures for a path,
the engine utilizes an \textit{automated theorem prover} whose purpose is to
check whether there is any violation of the property under analysis that
would lead to a failed execution, besides verifying path feasibility. A path
is said to be feasible in such scenario, if there exists an assignment of
concrete values to its symbolic encoding that would yield the satisfiability
of its boolean formula. Along with \textit{path-explosion} due to the exponential state
space exploration, non-deterministic inputs and several other vulnerabilities,
symbolic execution efficiency also relies to the use of efficient external
theorem provers. Ongoing research attempts to face those threats in order
to provide better tools that aid developers and help discover bugs in
programs.

\section{Automated Theorem Proving}

\textit{Automated theorem proving}, namely the ability to automatically prove a
mathematical theorem has been the insatiable desire of the scientific
community for an extensive period of time. Theorem proving's foundation
is essentially mathematical logic formulas and modelling. On the other hand,
\textit{Computer Science} foundations lie within mathematics and logic.
Many times software developers want to prove that a specific property is satisfied
for a program and its given input values, like its termination. Computer
programs can also be encoded as a logical formula and so formal theorems may be
constructed in the programming setting too.

The purpose of an automated theorem prover (ATP) is ideally to answer whether a
theorem can be proven or not, providing at the same time a formal proof
or a counterexample. Automated theorem proving was initially considered to be
part of Artificial Intelligence (AI). However, human assistance would be essential
in many cases, leading at the same time to the recognition of unable to be proven
theorems (such as termination).

Automated theorem proving has led to the creation of several tools following
different approaches to solve the same problem: proof construction. Of particular
interest have been satisfiability modulo theories (SMT) provers (or solvers)
that are the main tools employed in symbolic execution. SMT is a generalization
of the traditional boolean satisfiability problem (SAT) that utilizes underlying
decidable mathematical theories in order to encode and give semantical
meaning to function predicates instead of boolean (binary) variables. Such theories
are the theory of linear arithmetic, bitvectors and arrays and provide a powerful
expressiveness to the related solvers which could also be considered as
constraint satisfaction engines. Provided a predicate formula encoding, a prover
aims to check whether the formula is satisfiable; there is an assignment of values
to the compounded predicates. The state of the art tools for theorem proving
have been the Z3 Theorem Prover\cite{de2008z3}, and the Coq Proof
Assistant\cite{Coq}, an interactive theorem prover

\section{Datalog and \doop{}}

\subsection{Datalog}

\textit{Datalog} is a declarative programming language, essentially a subset of the  \textit{Prolog}
programming language due to the lack of function symbols, the cut operator among
other features. It is also usually considered a query language and has been described
as pure SQL enhanhanced with recursion.

Clause ordering does not matter in Datalog, and thus any order of computation is
guaranteed to yield the same results, unlike Prolog. Computation in Datalog is
established by declaring \textit{input facts} along with \textit{reasoning rules}. Based on those,
anything that can be inferred is computed by the underlying engine reaching
a convergence state due to the fact that only new facts, that is asserted truth,
can be produced by the inference rules. Such rules are called \textit{monotonic} and
the associated computation is the \emph{fixed-point}: rules produce new facts
until a convergence state is reached.

The main statements of Datalog are the inference rules, usually consisting of
the \emph{head} and the \emph{body}, or formally \texttt{rule}. Both the head and the body
may consist of conjunctions (expressed with the comma operator ,) or
disjunctions (expressed with the semicolon operator) of predicates, equivalently called
relations. Sharing many similarities with databases, a relation can be thought
of as a database table, whilst input facts and inferred knowledge can be thought of as
the database rows. The comma operator along with the variable constraints that
may form along the operand relations describes what is essentially known
as the join operator in the database community.

Datalog has been extensively used both in the academic world and the industry. Several
prominent knowledge databases have been designed with Datalog being the underlying engine
(i.e. LogicBlox). Datalog also finds use cases in the construction of network
specification files, distributed systems, security analysis of
smart contracts\cite{grech2018madmax}
and static program analysis. One of the
static analysis tools that have successfully embodied Datalog to express its analyses
is the \doop{} framework.

\subsection{The \doop{} framework}

The \doop{} framework is a static program analysis tool particularly performing
pointer analysis for the languages of the Java Virtual Machine (JVM) ecosystem.
Late work has also included analysis of Python, and especially TensorFlow programs.

\doop{} analyses are expressed in the Datalog language in a purely declarative
setting and the framework itself is known to be the first to introduce full,
end-to-end context-sensitive analyses for JVM based languages. The use of
Datalog has lead to the generation of new algorithms for pointer analysis
that manage to scale, yet provide accurate results.

\doop{} at first utilizes the Soot framework to convert the initial
program, usually compressed as a Java ARchive (JAR) into Java bytecode.
Based on the bytecode version of the program, \doop{} generates the input
facts for the analysis, introducing several relations that can be pre-computed.
The framework then includes the Datalog based Souffle tool for
static analysis\cite{jordan2016souffle} in order to perform any reasoning starting from
the provided input facts. That is, Souffle is the tool in which any reasoning happens.
\doop{} analyses have been expressed by a bunch of Datalog rules in Souffle Datalog.
Even though several context-sensitive analyses have been expressed, there is not any
path-sensitive analysis within the framework, thus it relies on the flow-sensitivity
introduced by Soot.
