\chapter{EVALUATIONS}\label{c:evaluations}

In this chapter we provide a short evaluation of
our approach. The main purpose of this work was to investigate
how easily could a theorem prover be implemented in the Datalog
programming language. Our main motivation for this
work has been the successful application of theorem provers
in techniques such as symbolic execution. To this end, we
wanted our approach to lay the foundations for the integration
of a theorem prover in \doop{}'s reasoning. We evaluated our
approach for three different input JARs based on the DaCapo
benchmarks\cite{DaCapo:paper}. These JARs are associated to
ANTLR \cite{ANTLR}, HSQLDB \cite{HSQLDB} and Jython \cite{JYTHON}.
In the following table we provide some measurements regarding
the time needed for the analyses to run. We have evaluated
our approach on \doop{}'s context-insensitive analysis with
symbolic reasoning either turned on or off. For the symbolic
reasoning runs, we also wanted to evaluate how many program expressions
are identified along with how many expression implications are infered.

\begin{center}
\captionof{table}{Analysis evaluation}
\label{tab:analysis}
\begin{tabular}{ |c|c|c|c|c|c| }
 \hline
 Input Jar & Fact Generation & CI + Symbolic & CI & Expressions & ExprImpliesOther\\
 \hline
 ANTLR & 49 sec & 48 sec& 33 sec & 608.547 & 863.864 \\
 \hline
 HSQLDB & 51 sec& 43 sec& 25 sec& 653.207 & 916.722 \\
 \hline
 Jython & 33 sec& 79 sec& 77 sec& 418.389 & 590.549 \\
 \hline
\end{tabular}
\end{center}

\doop{}'s context-insensitive analysis performs a pointer analysis
without any context consideration for its reasoning. Our symbolic
reasoning approach is implemented as complement to any of \doop{} main analyses, and
thus it further burdens the analysis execution times, as observed
in table \ref{tab:analysis}. It is also worth mentioning that \doop{} does not
benefit in any way from our approach at the moment.
However we may easily notice that our reasoner, even
though quite simple, manages to identify a set of expressions, also
providing a relatively satisfying number of expression implications.
These numbers are quite encouraging, as further restricting the
reasoning to discard certain sets of expressions and introducing
more logic theories as part of our reasoner would certainly provide
more accurate results.
