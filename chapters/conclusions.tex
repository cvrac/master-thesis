\chapter{CONCLUSIONS AND FUTURE WORK}\label{c:conclusions}

In the previous chapter we provided a brief evaluation of our
approach. By introducing our reasoner as part of \doop{}, we
managed to demonstrate several encouraging results. We have also
managed to demonstrate the simplicity and expressiveness of
a declarative language such as Datalog. The implementation
of both core reasoner and its preliminaries came quite naturaly
to us, proving Datalog to be a tool to be considered for such
applications. At the same time we also enhanced \doop{} with
several constructs such as the expression type which could
prove valuable to \doop{} for further usage.

Our work opens a plethora of research directions to investigate
in the context of \doop{} and declarative static analysis in
general. For example, it would make sense to explore the case
of completely integrating the reasoner in a pointer analysis
performed by \doop{}. Such an investigation would require the
utilization of the introduced path-predicates and inference
rules. The path-predicates are essentially expressions that
encode the control-flow constructs of a program, and thus
inferencing over them would lead to the identification of
unreached program locations not to be included during a
pointer analysis reasoning. As of the moment \doop{} completely
lacks path-sensitivity, thus we believe that it would probably
be of value to research towards this direction.
