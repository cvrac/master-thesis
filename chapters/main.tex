\chapter{Static Declarative Symbolic Reasoning}\label{c:main-content}

In this chapter we present our approach towards static declarative
symbolic reasoning. The main core of the reasoning is encoded as a
bunch of Souffle Datalog rules, along with the related input facts.
\doop{} includes a whole of preprocessing, expressive analyses and
postprocessing pieces that can be leveraged to enhance the core of
our reasoning at relatively swift speed.

\section{Schema Relations and Program Expression Trees}

\subsection{Schema}

\doop{} includes a set of fact generators that provide us with an
extensive set of input facts that may be leveraged for our reasoning.
Those generators either target different input languages or different
intermediate representations (IR). As an example, there are generators
that target the Java language, Android technology or Python. Those
generators may further utilize different tools that would take care
the translation of the input program into the corresponding IR; the
current tools being used by \doop{} are namely Soot and WALA. In the case of
Java, the generators ought to produce the same set of input facts
that \doop{} consumes for its main analyses, and thus there is a set
of files that define the common predicates-facts that should be
produced by the generators. The existing set of generated
facts is quite extensive and it is constantly further enriched, yet it
does not provide us with a set of facts crucial for the reasoning
we introduce. Hosting a rich whole of various pointer analyses,
\doop{} mainly needs knowledge that is related to pointers, or
in the Java setting reference types. As such, \doop{} lacks
presence of multiple facts in regard to primitive types or
values and in the case that they exist they are relatively
non-adequate. In that notion we decided to modify the common fact
generator in a way that enhances \doop{} with facts to be consumed
by our symbolic reasoning approach, or modifies several existing ones.
At the same time those facts may also come handy for \doop{} itself
in the future. In order to be able to construct a fine-grained
analysis we modified a certain amount of facts as follows in figure \ref{fig:facts-modified}.

\begin{figure}[th]
\rel{\_AssignOperFrom}{instruction, position, from}.\\
\rel{\_IfVar}{instruction, position, var}.
\captionof{figure}{Modified \doop{} facts}
\label{fig:facts-modified}
\end{figure}

\_AssignOperFrom introduces a new position variable that keeps track of
each subsequent operand of an instruction's right-hand side (rhs), whilst
the same also holds for \_IfVar too. Next, we provide a glimpse of all the
newly introduced generated facts in figure \ref{fig:facts-new}.

\begin{figure}[th]
  \rel{\_OperatorAt}{instruction, operator}.\\
  \rel{\_DummyIfVar}{instruction, var}.\\
  \rel{\_AssignOperFromConstant}{instruction, position, from}.\\
  \rel{\_IfConstant}{instruction, position, constant}.
  \captionof{figure}{New \doop{} facts}
  \label{fig:facts-new}
\end{figure}

Those input facts share many similarities in regards to the associated
variables - relation columns - though each one encodes crucial knowledge.
The \_OperatorAt relation holds facts that encode the operation being present
at a given instruction. \_AssignOperFromConstant is an alternate rule of \_AssignOperFrom,
whilst \_IfConstant is equivalent to \_IfVar. A constant type is different to a variable type
in the context of the framework, thus the need for a more detailed differentiation between the
two. Such differentiations have extensively been made during the development of our approach, but
in many cases will be omitted in the context of this text.
Lastly, the \_DummyIfVar relation has been introduced in order to encode the conditional
expressions at a given branch instruction, that is quite necessary to our implementation
and its need is evoked due to the IR provide by Soot. It is also worth mentioning that
the facts and the associated relations as described above are provided by the generators
written in Java. Mere preprocessing takes place before importing the final facts into \doop{}'s
knowledge base in order to follow its conventions. That preprocessing leads to the following
renames as described in figure \ref{fig:imported-facts}.

\begin{figure}[th]
  \rel{AssignOper\_From}{instruction, position, from}.\\
  \rel{AssignOper\_FromConstant}{instruction, position, from}.\\
  \rel{If\_Var}{instruction, position, var}.\\
  \rel{If\_Constant}{instruction, position, constant}.\\
  \rel{DummyIf\_Var}{instruction, var}.\\
  \rel{Operator\_At}{instruction, operator}.\\
  \captionof{figure}{Imported facts}
  \label{fig:imported-facts}
\end{figure}

The relations described thus far are only a small set of the input to our reasoning, as we have
made use of the various relations and rules that were already existed in \doop{}. Here we have
only presented the brand new input relations or the ones that we modified. However we can
not provide an extensive presentation of all, thus we will briefly describe them in the
following sections, in the context of the rules that they are used.

\subsection{Program Expression Trees}

Our symbolic reasoning approach namely reasons about a program's possible expressions. That
said, there is a need for the representation of a program's expression, either primitive ones
or not. \doop{} does not include such an encoding, due to the fact that its analyses mainly
reason about pointers. In order to address that and also lay the foundations of our approach,
we have introduced the following types as described in figure \ref{fig:expr-type}.

\begin{figure}[th]
  \type{SymbolicInput}\singlefield{Var | MethodInvocation | NumConstant}\\
  \type{Operator}\singlefield{symbol}\\
  \type{ExpressionType}\singlefield{PrimitiveType | ReferenceType}\\
  \type{Base}\singlefield{SymbolicInput | Operator}\\
  \type{Expr}\typefields{\tab base: Base}{\tab type: ExpressionType}{\tab left: Expr}{\tab right: Expr}\\
  \captionof{figure}{Expression Type}
  \label{fig:expr-type}
\end{figure}

The types that are defined using a (\dl{|}) symbol are said to inherit from the
associated types, whilst the latter definition introduces a new record type. The
expression type manages to represent both unary, binary and constant expressions at the same time,
by distinguishing each case with the help of the \textit{base} field. If base is of type
\textit{SymbolicInput} then the latter \textit{Expr} fields are nil, whilst in the case of
base being of \textit{Operator} at least the first of the Expr fields is indeed an expression.
For each distinct expression we also preserve its type, which can either be
\textit{PrimitiveType} or \textit{ReferenceType}. Those types are the building blocks for the
construction of the program's expression. It may come natural to the reader that
the encoding leads to complex expression trees.

\section{Path Expressions}


\section{Boolean Symbolic Reasoning}


\subsection{Propositional Logic Axioms}


\subsection{Propositional Logic Inference Rules}
